\documentclass{book}
\usepackage[T1]{fontenc}
\usepackage[french]{babel}
\usepackage{latexsym}
\usepackage{amsmath,amsfonts,amsthm,amssymb}
\usepackage{arydshln}
\usepackage{dsfont}
\usepackage{enumitem}
\usepackage{todonotes}

\usepackage{minitoc}

\usepackage{xcolor}

\newtheoremstyle{own}%
    {3pt}% Space above
    {3pt}% Space below
    {}% Body font
    {}% Indent amount
    {\color{blue}\bfseries}% Theorem head font
    {:}% Punctuation after theorem head
    {\newline}% Space after theorem head
    {}% Theorem head spec

\theoremstyle{own}
\newtheorem{example}{Example}[section]

\theoremstyle{definition}
\newtheorem{definition}{Definition}[section]

\theoremstyle{remark}
\newtheorem*{remark}{Remarque}

\newtheorem{prop}{Proposition}[section]

\newtheorem{theorem}{Theorem}[section]

\usepackage{tikz}
\newcommand*\circled[1]{\tikz[baseline=(char.base)]{
            \node[shape=circle,draw,inner sep=2pt] (char) {#1};}}

\renewcommand\qedsymbol{$\blacksquare$}

\begin{document}

\frontmatter
\dominitoc
\tableofcontents

\todo{Retravailler les commande remarque et definition}\\

\mainmatter
\chapter{Introduction}
\minitoc

	\section{Virgule flottante normalis�e}
		\begin{definition}
		IEEE754 \\
		Soit $x \in \mathds{R}$. on fit que $x$ est un virgule flottante normalis�e \underline{de base $b$ avec mantisse � $m$ chiffres} \\
		si:
			\begin{enumerate}[label=\alph*]
				\item $b \in \mathds{N} \setminus \{0,1\}$
				\item $x = \pm 0,a_1, a_2, \dots, a_m * b^M$ pour un $M \in \mathds{Z}$ et $a_i \in\{0,1,2,\dots. b-1\}$
				\item $a_1 \neq 0$
			\end{enumerate}
		\end{definition}
	
		\begin{example}
			\begin{enumerate}[label={\alph*)}]
				\item
					\begin{tabular}{l}
						$2,35 \leadsto +0,235 \times 10^1$\\
						$-0,00235 \leadsto -0,235 \times 10^{-2}$
					\end{tabular}
							
				\item
					\begin{tabular}{rl}
						$(13)_{10}$ & $= 2^3 + 2^2 + 0\times 2^1 + 2^0$\\
						& $=(1 1 0 1 )_2$\\
						$(42)_{10}$ & $=2^5 + 0\times2^4 + 2^3 + 0\times 2^2 + 2^1 + 0 \times 2^0$\\
						& $(1 0 1 0 1 0 )_2$
					\end{tabular}\\
					\begin{tabular}{r|l}
						420\\
						\hline
						210 & 0 \\
						105 & 0 \\
						52 & 1 \\
						26 & 0 \\
						13 & 0 \\
						6 & 1 \\
						3 & 0 \\
						1 & 1 \\
						0 & 1
					\end{tabular} \Bigg \uparrow $(420)_{10} = (110100100)_2$ \\ \\
					$(0,5)_{10} = 1\times 2^{-1} = (0,1)_2$\\ \\
					$(0,625)_{10} = 1\times 2^{-1} + 0\times 2^{-2} + 1\times 2^{-3} = (0,101)_2$\\ \\
	
					\begin{tabular}{cc}
						\Bigg \downarrow &
						\begin{tabular}{r|l}
							$0$ & $,171875$\\
							\hline
							$0$ & $,34375$\\
							$0$ & $,6875$\\
							$1$ & $,375$\\
							$0$ & $,75$\\
							$1$ & $,5$\\
							$1$ & $,0$\\
						\end{tabular}% &
						\begin{tabular}{rl}
						$(0.171875)_{10}$ & $= (0,001011)_2$\\
						& $=(+0,1011\times 2^{-2})_2$
						\end{tabular}
					\end{tabular}\\
					
					\begin{tabular}{rl}
						$(1/3)_{10}$ & $=\frac{1}{4} + \left(\frac{1}{3} - \frac{1}{4}\right)$\\
						& $=\frac{1}{4} + \frac{1}{12}$\\
						& $=\frac{1}{4} + \frac{1}{16} + \left( \frac{1}{12} - \frac{1}{16} \right)$\\
						& $=\frac{1}{4} + \frac{1}{16} + \frac{1}{48}$\\
						& $=\frac{1}{4} + \frac{1}{16} + \frac{1}{64} + \frac{1}{192}$\\
						& $=(+0,10 \times 2^{-1})_2$\\
					\end{tabular}\\
					\todo{demander a mario des explications pour celle-la}
					
				\item En base $7$\\
					$(500,34)_{10}$\\
					\begin{tabular}{ccc}
						\begin{tabular}{r|l}
							$500$ \\
							$71$ & $3$ \\
							$10$ & $1$ \\
							$1$ & $3$ \\
							$0$ & $1$
						\end{tabular}
						& \Bigg \uparrow
						& $(500)_{10} \rightarrow (1 3 1 3)_7$
					\end{tabular} \\
					\todo{ajouter les choses qui manque au dessus de (...)} \\ \\
					\begin{tabular}{ccc}
						\Bigg \downarrow &
						\begin{tabular}{r|l}
							$0$ & $,34$ \\
							$2$ & $,38$ \\
							$2$ & $,66$ \\
							$4$ & $,62$ \\
							$4$ & $,32$ \\
							\hdashline
							$2$ & $,38$ \\
							$2$ & $,66$
						\end{tabular} &
						$(0,34)_{10} = (0,\overline{2244})_7$
					\end{tabular}\\
					$(500,34)_{34} = (1313,\overline{2244})_7$
					
				\item En base $16$\\
					$(670)_{10}$\\
					\begin{tabular}{ccc}
						\begin{tabular}{r|l}
							$670$ \\
							\hline
							$41$ & $14$ \\
							$2$ & $9$ \\
							$0$ & $0$
						\end{tabular} &
						\Bigg \uparrow &
						$(670)_{10} = (29E)_{16}$
					\end{tabular} \\ \\
					
					\begin{tabular}{rl}
						$(A3F,2D)_{16}$ & $=10 \times 16^2 + 3 \times 16^1 + 15 \times 16^0 + 2\times 16^{-1} + 13 \times 16^{-2}$\\
						& $=2623,17578125$
					\end{tabular}
			\end{enumerate}
		\end{example}
		
		\begin{prop}
			Si $M_{min} \leq M \leq M_{max}$, alors la quantit� de nombres r�els repr�sentables en virgule flottante normalis� de base $b$ � $m$ chiffres est $2(b-1)b^{m-1}(M_{max}-M_{min}+1)+1$
		\end{prop}
		\begin{proof}
		$\pm asdfasdf$
		\end{proof} \todo{ajouter la preuve}	
	
		\subsection{Op�rations arthm�tique}
			 $b=10$, $m=3$, $M_{min}=-15$, $M_{max}=16$ \\
			\begin{itemize}			
				\item{\underline{Addition}}\\				 
					 \begin{tabular}{rl}
					 	$(+0,125\times 10^6)_{10} + (-0,128\times 10^6)_{10}$ & $=(-0,003 \times 10^6)$ \\
					 	& $=(-0,3 \times 10^4)_10$
					 \end{tabular} \\ \\
					 
					$x=(+0,125\times 10^6)_{10},y=(+0,437\times 10^{12})_{10}$ \\ \\
					\begin{tabular}{rl}
						$x+y$ & $=(+0,125\times 10^6)_{10}+(+0,437\times 10^{12})_{10}$ \\
						& $=(+0,000000125\times 10^12)_{10}+(+0,437\times 10^{12})_{10}$ \\
						& $=(+0,437000125\times 10^{12})_{10}$ \\
						& $=(+0,437\times 10^{12})_{10}$
					 \end{tabular}\\
					 \\
					 donc $x+y=y$ m�me si $x \neq 0$.
					 
				\item{\underline{Multiplication}}\\
					$(+0,125 \times 10^{6}) \times (+0,437 \times 10^{12})$ \todo{add indication note papier} \\
					\\
					$=+0,054625 \times 10^{18}$
					$=(0,54625 \times 10^{17})$
					$=(0,546 \times 10^{17})$
			 \end{itemize}
			 
			 \begin{prop}
			 	Soient $x,y$ � $m$ chiffres en base $b$. Si $\frac{|x|}{|y|} \leq b^{-m-1}$ alors $x+y = y$ \todo{refaire mise en page}
			 \end{prop}
			 
			 \begin{proof}
			 	asdfasdfasdf
			 \end{proof} \todo{ajouter la preuve}
			 
			 \begin{itemize}
			 	\item{\underline{R�gles d'arrondi}} \\
			 		$m=3$ \\ 
			 		\\
			 		\begin{tabular}{rl}
			 			$0,23576$ & $\implies 0,236$\\
						$0,233524$ & $\implies 0,235$\\
						$0,23509$ & $\implies 0,235$\\
						$0,2355$ & $\implies 0,236$\\
						$0,2345$ & $\implies 0,235$
			 		\end{tabular}
			 		\\
			 		Si se termine par $5$, on arrondit pour que le dernier chiffre soit pair.\todo{reformuler et remettre en page}
			 \end{itemize}
			 
			 
	\section{Comparaison th�orie/pratique}
		\todo{a completer}
		mettre qqc ici
%		Avant d'entreprendre l'�tude de probl�mes d'optimisation il est bien de d�finir ce qu'est un probl�me d'optimisation.\\
%
%Celui-ci consiste a d�terminer la valeur possible qu'une fonction r�elle $$f\colon E \to \mathbb{R}$$ nomm� fonction objectif, peut prendre dans l'ensemble $E$, nomm� ensemble r�alisable.\\
%
%\begin{itemize}
%	\item Pour la minimisation
%		$$
%			f^* = \inf_{x \in E} f(x)
%		$$
%
%		Cela signifie que
%		\begin{enumerate}
%			\item $f(x) \geq f^* \text{~~~~}\forall x \in E$
%			\item $\forall \epsilon > 0, \exists x_{\epsilon} \in E \mid f(x_{\epsilon}) < f^* + \epsilon$
%			\begin{example}
%				Soit $E=\mathbb{R}$ \\
%				Si $f(x) = e^x$, on a que
%				$$0 = \inf_{x \in E} f(x), \text{ mais } f(x) > 0 \text{~~} \forall \in E$$
%
%				Par contre si $f(x)=x^2$ on a que\\
%				$$0 = f(0) = \inf_{x \in E} f(x) = \min_{x \in E} f(x)$$
%				
%			\end{example}
%		\end{enumerate}
%
%	\item Pour la maximisation
%		$$
%			f^* = \sup_{x \in E} f(x)
%		$$
%
%		\begin{enumerate}
%			\item $f(x) \leq f^* \text{~~~~} \forall x \ in E$
%			\item $\forall \epsilon > 0, \exists x_{\epsilon} \in E \mid f(x_\epsilon) > f^*-\epsilon $
%		\end{enumerate}
%\end{itemize}
%
%Notons que maximiser $f$ revient � minimiser $-f$. Ainsi sans perte de g�n�ralit�, on peut uniquement consid�rer les probl�mes de minimisation.\\
%
%Sans ajouter d'hypoth�se sur la fonction $f$ et l'ensemble $E$, il n'est pas certain que l'on puisse trouver un �l�ment $x^*$ tel que $f(x^*) = f^*$. Lorsque c'est le cas, la formulation math�matique devient:
%
%$$
%	f(x^*)=f^*=\min_{x\in E} f(x)
%$$
%
%De mani�re �vidente, les probl�mes pour lesquels il existe un �l�ment de l'ensemble $E$ tel que $f(x^*)=f^*$ seront particuliairement int�ressants. Un tel pointe de l'ensemble $E$ est nomm� optimum et il peut �tre un minimum ou un maximum.
%
%\section{Types d'optimums}
%	Sans hypoth�se additionnelles, les probl�mes tels que d�crits pr�c�demment sont en g�n�ral impossible � r�soudre. Consid�rons la fig. 1
%
%\todo{Ajouter la fig.1}
%	\begin{itemize}
%		\item O� se situe le minimum de la fonction?
%		\item Comment identifier lequel des nombreux (voir infinis!) miminum apparant est le plus petit?
%	\end{itemize}
%
%	Pour arriver � �tudier les probl�mes d'optimisation nous allons classifier les optimums selons diff�rents crit�res.
%
%	\subsection{Optimums locaux et globaux}
%		Jusqu'� maintenant, nous avons d�fini les optimums en comparant la valeur de la fonction $f$ � l'optimum avec sa valeur en tout autre point de $E$. Ce type de probl�me est connu sous le nom d'optimisation globale. Si on baisse un peu les attentes et que l'on compare les valeurs de la fonction $f$ dans un voisinage, alors $x^*$ est un minimum local s'il existe $$\epsilon > 0 \text{ t.q. } f(x) \geq f(x^*), \text{  }\forall x \in E \cap V_\epsilon(x^*)$$
%qui est un voisinage de diam�tre $\epsilon$ centr� en $x^*$.
%
%	\subsection{Optimums stricts}
%		La notion d'optimum strict conserne le fait que d'autre point du voisinage ne puissent avoir la m�me valeur que $x^*$. Par exemple � la figure 2, les zones encercl�es contiennent des optimums qui ne sont pas stricts, tandis que les autres le sont.
%
%		\todo{Ajouter la fig. 2}
%
%		\begin{definition}
%			Un minimum local $x^*$ est dit strict s'il existe une valeur $\epsilon > 0 \text{ t.q. } \forall x \in V_\epsilon(x^*)$ et pour $x\neq x^*$ $f(x)>f(x^*)$.
%		\end{definition}
%		\todo{Ajouter underline dans la d�finition}
%
%	\subsection{Optimums isol�s}
%		Sur la figure 2, les optimums encercl�s ne sont pas stricts, mais chaque ensemble d'optimums est s�par� les uns des autres. La notion d'optimum isoll� formalise cette situation.
%
%		\begin{definition}
%			Un ensemble connexe d'optimums $O$ est dit isol� si quelque soit un optimum $x \not\in O$, la distance de $x$ � $O$ est born� inf�rieurement par une constante positive.
%		\end{definition}
%		\todo{Ajouter underline dans la d�finition}
%
%		\begin{example}
%			La fonction $\left(x sin\left(\frac{1}{x}\right)\right)^2$ comporte une accumulation de minimum locaux � l'origine. L'origine n'est pas un optimum isol� ni strict, car une infinit� de points valent $0$ pr�s de l'origine.
%		\end{example}
%
%\section{Conditions d'optimalit�}
%\todo{v�rifier l'orthographe du titre de la section}
%	Malgr� les classifications pr�c�dentes, il demeure difficile d'identifier ou de v�rifier qu'un point $x^*$ est un optimum d'un probl�me . En effet, les d�finitions dont nous disposons exigeraient de comparer les valeurs d'une fonction en un point $x^*$ avec sa valeur en une infinit� de points voisins de $x^*$. L'analyse math�matique vient alors � notre secours.
%
%	\subsection{Conditions pour un point stationnaire}
%		On apprend dans les permiers cours de calcul qu'une fonction poss�de des optimums locaux en des points qui annulent sa d�riv�e. De tels points sont nomm�s point stationnaires. Les points stationnaires pour une fonction r�elle peuvent �tre de trois types: minimums locaux, maximum locaux et points d'inflexions.\\
%
%Par cons�quent, tout minimum local est un point stationnaire, mais l'inverse n'est pas vrai. Les points stationnaires satisfont donc a la condition n�cessaire d'optimimalit� de premier ordre: $f'(x)=0$
%	\subsection{Conditions pour un optimum}
%		On apprend aussi dans les cours de calcul que si,  en point stationnaire, la d�riv� seconde d'une fonction est positive il s'agit d'un minimum local. Si elle est n�gative, il s'agit d'un maximum local et si elle est nulle one ne peut rien conclure. Les points stationnaires qui satisfons � la condition suffisante d'optimalit� de second ordre: $f''(x)>0$ sont donc des minimums.
%
%		\begin{theorem}
%			Soit $f$ une fonction de classe $C^2$. Un point $x^*$ qui satisfait aux conditions n�cessaire et suffisante d'optimalit� est un minimum local strict et isol�.
%		\end{theorem}
%
%		Le cas $f''(x^*)=0$ comporte une ambiguit� que l'on peut illustrer par les fonctions $$f_1(x)=x^4 \text{ et } f_2(x)=-x^4$$
%
%		en $x^*=0$, pour les deux fonctions, on a que 
%$$f'(0)=f''(0)=0$$
%
%bien que l'origine soit un minimum pour $f_1$ et un maximum pour $f_2$.
%\include{chap2/chap2}
%\include{chap3/chap3}

\backmatter
	
\end{document}